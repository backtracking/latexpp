\documentclass[compress]{beamer}
\usepackage[latin1]{inputenc}

\begin{document}

\begin{frame}\frametitle{tt}

environnement \texttt{tt} :
\begin{tt}
#tagada         #
#une autre ligne#
\end{tt}

environment \texttt{lightblue-tt}
\begin{lightblue-tt}
module M = struct
  type          t = int -> int * int           a
  and           u = int                       aa
  toto :  titi
  toto::  tutu
  toto :: tutu
aaaaaaaaaaaaaaaaaaaaaaaaaaaaaaaaaaaaaaaaaaaaaaaa
\end{lightblue-tt}

caract�res sp�ciaux
\begin{lightblue-tt}
AAAAAAAAAAAAAAAAAAAAAAAAAAAAAAAAAAAAAAAAAAAAAAA
### $$$ \\\ ::: ___ %%% ~~~ ;;; &&& ^^^ {{{ }}}
\end{lightblue-tt}%$
\end{frame}

\begin{frame}\frametitle{ocaml}
Environment \texttt{ocaml}
\begin{ocaml}
  type t = int -> int (* def of type t { ... } *)
  let v = { x = 1 ; y = 2 } and w = ...
  let rec f x = if x <= 1 then 1 else x * f (x-1)
\end{ocaml}
and environment \texttt{ocaml-tt}
\begin{ocaml-tt}
  type t = int -> int
  type 'a u = A of 'a * 'a 
\end{ocaml-tt}
and environment \texttt{ocaml-sf}
\begin{ocaml-sf}
  type t = int -> int (* def of type t *)
  let v = { x = 1 ; y = 2 } and w = ...
  let rec f x = if x <= 1 then 1 else x * f (x-1)
\end{ocaml-sf}
\end{frame}

\begin{frame}\frametitle{Cha�nes de caract�res}
\begin{ocaml}
  let s = "toto"
  let s = "a (* hop *) b"
  let c = 'a'
  let c = '\\' 
  let c = '\123'
  let c = '\n'
  let c = '"'
  let s = "abc\"def"
  let s = "abc\n\t\123\\" (* abc *)
  let s = "abc"
\end{ocaml}
\end{frame}

\begin{frame}\frametitle{Ocamllex}
\begin{ocamllex}
let x = ['a'-'z']    
rule f = parse
| "(*" { action } (* commentaire *)
| "*)" { autre action } 
\end{ocamllex}
\end{frame}

\begin{frame}\frametitle{mips}
avant
\begin{mips}
    move $a0, 10  
    add  $a0, $a1
\end{mips}
apr�s
\begin{mips}
      .text
  L:  addi $sp, -4      # et un commentaire\
      li   $a0, 42      # et un autre { et } 
      .data
#######################
\end{mips}

avant
\begin{mips-tt}[vspacing=""]
  addi $sp, 4
\end{mips-tt}
apr�s
\end{frame}

\begin{frame}\frametitle{pascal}
avant
\begin{pascal-lightblue-tt}
program fib;
var
    x : integer;
begin
    x := 0;
end. (* et voil� $ & # *)
\end{pascal-lightblue-tt}
apr�s
\begin{pascal-tt}
procedure f(var x : integer);
begin
  x := 0;
end;
\end{pascal-tt}
et encore apr�s
\end{frame}

\end{document}

%%% Local Variables: 
%%% mode: latex
%%% TeX-master: t
%%% End: 
